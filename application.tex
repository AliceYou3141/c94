\chapter{TLSとアプリケーション層}

TLSがどのように通信を行っているか、どのようにTCPというサービスを利用しているかは、ここまでで説明をしました。では、TLSを利用するアプリケーションからは、TLSはどのように見えているのでしょうか。そして、どのようにTLSを利用するのでしょうか。この章では、TLSを利用するアプリケーションの画家から、TLSをみていくことにします。

\section{TLSを使用したアプリケーションの通信}

アプリケーション層からTLSを見ると、どのように見えているのでしょうか。これは、TLSの昔の名前であるSSLのほうがわかりやすいでしょう。Secure Socker Layerというその名の通り、socket(2)にかぶせるラッパとして実装されています。では、アプリケーションをTLSに対応させるというのは、どういうことなのでしょうか。

TLSに対応した最初のアプリケーションプロトコルは、HTTPでした。もっと正確に表現するならば、TLSの前身であるSSLは、HTTPの通信を保護するために開発されています。
TLSにおけるApplication DataがRecord層にアプリケーションからのストリームをそのまま渡すのは、いわばその名残です。そのため、初期のSSLの解説では、Application Dataの部分がそのままHTTPになっているものがありました。

TLSを使用するということは、アプリケーション層における通信が、トランスポート層に渡る前に暗号化されるということです。
これは、TLSを使用しない従来のアプリケーションとは、アプリケーションレベルの通信の互換性が失われるということでもあります。
では、TLSを使用しない従来のアプリケーションと互換性を確保しつつ、TLSを使用する方法には、どのようなものがあるのでしょうか。

そのやり方には、二つの方法があります。ひとつは、従来、そのアプリケーションが使用しているポートとは別のポート番号を、TLSによる通信専用に使用する方法です。もう一つは、TCPをサービスとして利用する通信を始めてから、TLSを使用可能であるかを通信相手に問い合わせて、可能であればTLSを使用した通信に移行するという方法です。
この二つの方式の、前者を暗示的TLS(Implicit TLS)、後者を明示的TLS(Explicit TLS)とよんでいます。

\subsection{暗示的TLS}

\begin{figure}[htbp]
	\includegraphics[width=10cm,clip]{draw/fig11.eps}
	\caption{TLSで別のポートを使用する場合}
	\label{fig:other_port}
\end{figure}


あるアプリケーションがTLSによる通信と、平文による通信とを使い分けるには、TLSを使用する通信と、TLSを使用しない通信で、使用するトランスポート層のポート番号を別にする方法があります。
このように、TLSを使う通信で使うオートと、平文で通信するときに使うポートを分ける方法を、暗示的TLS(Implicit TLS)とよびます。
暗示的TLSと呼ぶのは、TLSで通信をするポートでは、特に宣言など無く、暗黙の内にTLSによる通信を行うためです。
また、次に説明するStartTLSと対比して、SSL/TLSという表現をしている場合もあります。

TLSの前身であるSSLを利用した最初のアプリケーションプロトコルであるHTTPは、、ウエルノウンなポートとして80/TCPを使用します。
HTTPでTLSを使用する場合、その通信には、80/TCPとは別のポート番号である、443/TCPを使用するのが一般的です。このとき、プロトコル名はHTTPS(HTTP Secure)となります。

HTTP以外のプロトコルで、TLSによる通信に別のポート番号を用いる場合、、プロトコル名も、HTTPとHTTPSの関係に習って、プロトコル名の末尾にS(Secure)を追加します。たとえば、SMTPの通信でTLSを使用すればSMTPSであり、SMTPSにはポート番号として465/TCPが割り当てられています。

この方法の利点は、TLSを使用しない通信を行う場合は、従来のポートへこれまで通りアクセスすればよいことです。そして、TLSを使用する場合は、Secureとついたプロトコルに割り当てられたポートにアクセスします。
Secureのついたプロトコルのためのポートでの通信は、最初からTLSを通すようにすれば良い、という実装上のわかりやすさもあります。
また、TLSによる暗号化の有無でポートが別になっているので、サーバやネットワークの管理的な意味でも、わかりやすくなります。

欠点は、一つのアプリケーションに対して、それが使用するポート数の倍の数のポートを割り当てる必要があることです。このような書き方をしたのは、FTPなど、制御とデータ転送とで二つのポートを使用するプロトコルがあり、それもSedcureとしてTLSを介することがあるからです。

\subsection{明示的TLS}

\begin{figure}[htbp]
	\includegraphics[width=10cm,clip]{draw/fig12.eps}
	\caption{StartTLS}
	\label{fig:starttls}
\end{figure}


\begin{figure}[htbp]
	\includegraphics[width=10cm,clip]{draw/fig13.eps}
	\caption{StartTLSのタイムライン}
	\label{fig:starttls_timeline}
\end{figure}

アプリケーションの通信で、TLSを使う場合と使わない場合で、共通のポート番号を使用する方法があります。
それは、図\ref{fig:starttls}のように、平文でアプリケーション間の通信を始めてから、通信相手がTLSを使用できることが確認できたら、TLSを介した通信に移行するとう方法です。
もし、通信相手がTLSを使用しないなら、そのまま平文で通信します。この方法を、明示的TLS(Explicit TLS)、あるいは、TLSに移行する際のコマンド名からStartTLSと呼びます。そして、StartTLSを使用するアプリケーションのプロトコルは、プロトコル名の後に+StartTLSと付けて区別しています。

一番最初にRFCになったStartTLSである、SMTP+StartTLSでその手順を見てみることにしましょう。
StartTLSでは、最初の接続は、そのプロトコルに割り当てたら太ポートで、平文で行ないます。SMTP+StartTLSなら、通常のSMTPと同様に、25/TCP、または、Submission Portの587/TCPに接続します。
SMTP+StartTLSののタイムラインは、図\ref{fig:starttls_timeline}のように表されます。




%\begin{wrapfigure}[20]{r}{7cm}
%	\includegraphics[width=7cm, clip]{draw/fig13.eps}
%	\caption{StartTLSのタイムライン}
%	\label{fig:starttls_timeline}
%\end{wrapfigure}

SMTPであれば、SMTPコマンドのEHLOを送信したの後で、クライアントからサーバに、StartTLSというコマンドを送信します。
サーバ側がStartTLSコマンドを受け入れるなら、そのコネクションの中で、TLSのHandshakeを開始します。
TLSのハンドシェイクが成立したら、クライアントは改めて、TLSのApplication Dataとして、SMTPの通信開始をあらわすEHLOを送信します。そして、クライアントとサーバは、TLSの上でSMTPによる通信を行ないます。


\section{証明書の検証}
TLSでは、証明書の検証までは、その機能で行うことができます。ですが、その証明書の検証結果をもって、通信を続けるかどうかを判断するのはアプリケーションの役目です。

プロトコルの説明でも書きましたが、TLSは、あくまでも暗号化と検証が可能なTCPのラッパです。
TLSの機能で、証明書を検証した結果をもって、そこから先の通信を行うかどうか、という判断はアプリケーション、もしくは、ユーザが行う必要があります。
現在の多くのアプリケーションでは、TLSの証明書が、証明書チェーンで検証できない場合は、通信をしない設定がデフォルトとなっています。

\subsection{Webは全てHTTPになるべきか}

この本の以前の版である「SSLのはなし」では、なんでもかんでもSSLにすればいいというわけではない、というように書いていました。ですが、時代は変わり、Webの通信はTLSを使う、HTTPSが多くを占めるようになってきました。

これは、TLSの機能として、かつては暗号化が重視されていました。ですが、現在は、証明書チェーンによる認証が重視されるという状況の変化があります。Webであれば、そのWebサイトが正しいものなのか、アクセスした人に検証する材料を提供することで、信用を増す効果があります。
もちろん、通信をプライベートなものと考えて、通信の暗号化によってユーザのプライベート情報を保護するため、という目的もあります。

リアルタイム性を重視しないなら、TLSを使用したことによるオーバヘッドはそれほど問題になりません。そのため、正規のサーバにアクセスしないことになり得るTCPの通信は、TLSを利用した方が良いというのが現在の結論となります。

