\section*{謝辞}
\begin{center}
この本を読んでくださる方に \\
気力をくれる友人に \\
大切な人に \\
かわらぬ感謝と本書をささげます
\end{center}

\section*{まえがき}
TLSを利用したネットワークアプリケーションを書いたり、TLSを使用するサービスの運用をしている方は多いと思います。ですが、そのTLSがどのような仕組みで動いているか、そう問われるとどうでしょうか。もしその点を理解していなくても、TLSという、通信をセキュアにするおまじないを唱えることができます。

セキュリティでオレオレ証明はOKしちゃだめー、と、少しネットワークに詳しい人は言うでしょう。では、そのオレオレ証明とは、どんな証明書なのでしょうか。それ以前に、TLSで使用する証明書とは何なのでしょうか。TLSでは、RSAなどの公開鍵暗号を使っています。では、公開鍵暗号は、TLSを用いた通信のどこからどこまでを暗号化しているのでしょうか。
そして、TLSは、、トランスポート層としてTCPの使用を前提としています。ではなぜ、UDPという、コネクションレスの通信に対しては使うことができないのでしょうか。

正直なところ、これらのことを考えたこともない人も、多いのではないかと思います。本書は、TLSとは何をしているのか、何をしているから通信がセキュアになるのか、それについて説明していきます。

TLSは、TCPを利用するアプリケーションプロトコルとしてみれば、それほど難しくはありません。
ですが、そのプロトコルが持つ「意味」を理解しようとした瞬間に、途端に難しくなります。それは、暗号理論や公開鍵基盤という、プログラミングやネットワークの外にある知識が必要となるためです。
本書は、TLSが何故わかりにくいのかという観点で、イメージを掴むための解説を、重点的に行います。読者の皆様が、TLSをより理解するきっかけになれば幸いです。

\begin{flushright}
2018年4月21日 \\
インフラエンジニアの毒舌な妹(@infra\_imouto)
\end{flushright}

\section*{想定する読者と内容範囲}
TLSを使っている、TLSを使用するサービスを運用している、だけどTLSについてはいまいちよくわかっていない、というレベルの読者を想定しています。

TLSはプロトコルは、下位層としてのTCPに頼っている部分が多々あります。ですが、TCPに関しての説明は本書の範囲外として省略しています。必要であれば、TCP/IP二巻する各種解説書も併読してください。

また、本書は、ネットワークプログラミングやアプリケーション運用のために、TLSのイメージを明らかにすることを目的としています。
暗号化処理の数学的な説明、セキュリティに関する項目、特に、主な攻撃方法とTLSの対策などはについては省略しています。

\subsection*{上巻の内容}

\paragraph{第一章}
TLSの概要と、どのような基盤の上に成り立っているかを説明します。

\paragraph{第二章}
TLSで使用する暗号技術についての説明をします。

\paragraph{第三章}
TLSで使用するハッシュアルゴリズムについて説明します

\subsection*{下巻の内容}

\paragraph{第一章}
TLSで使用する証明書を成立させるための社会的な基盤についての説明をします。

\paragraph{第二章}
電子署名とその検証について説明をします。

\paragraph{第三章}
TLSのプロトコルについて、暗号やハッシュのアルゴリズム、証明書をどのように使って通信していくかの説明をします。

\paragraph{第四章}
TLSを利用するアプリケーションは、どのようにSSLを利用してプロセス間通信の暗号化を行うかについて説明します。

\paragraph{参考文献}
本書を執筆するのに参考にさせていただいた、書籍とWebページの一覧です。

\section*{免責事項}
本書に書いてあることは、筆者知識のレベルでまとめたものです。内容が正しいとは言い切れません。、これまで出した本でも相当やらかしています。また、学校のレポート、業務などのコードを書く際に、本書の内容を信じて書いて損害が生じても、筆者にその責任はありません。

くれぐれも、自己責任と十分な検証の上、ご利用ください。

\section*{表紙イラスト}
ゆうちゃん (コース英知)