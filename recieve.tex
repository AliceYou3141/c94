\chapter{他のサーバからのメールを受信する}

\section{メール転送を受けるための設定}

他のサーバからメールを転送してもらうというのは、ホストがメールサーバとして動作するということです。メールサーバの基本は、自分の管理するメールボックス宛のメールは受信し、そうでないなら適切なメールサーバに転送する、というものです。

ですが、外部から受け取ったメールが自分の管理するメールボックス宛でなかったからと言って、それをまた転送していては、それは迷惑メールを送信するのに使われてしまします。そのため、他のメールサーバからノメールを受け馬絵に、設定を見直しておきましょう。

\subsection{どこ宛のメールを中継するのかを決める}
このメールサーバは、どこ宛のメールを受け取るのでしょうか。その設定を行うmain.cf(5)のディレクティブとして、relay\_domainsを説明していました。このディレクティブは、性格医は、どこのドメイン宛のメールを中継するか、とうディレクティブです。

ここで言う中継とは、他のサーバに渡す、という意味だけで無く、このメールサーバのメールボックスへメールを入れるプロセスへの中継、という意味も含んでいます。そのため、もう一度relay\_domainsの設定値をチェックしておきましょう。

\begin{lstlisting}[basicstyle=\ttfamily\footnotesize, frame=single]
relay_comains = $mydestination
\end{lstlisting}

この設定例のように、\$relay\_domainsの値として\$mydestinationを使用している場合は、そちらも見直しておきましょう。もし、このメールサーバをドメイン宛のメールを受け取るメールサーバにするなら、\$mydestinationに\$mydomainを含むように、以下のように設定します。

\begin{lstlisting}[basicstyle=\ttfamily\footnotesize, frame=single]
# ドメインのメールサーバにするなら2行目か状況によっては
# 一番下の例を有効にする
#mydestination = $myhostname, localhost.$mydomain, localhost
mydestination = $myhostname, localhost.$mydomain,
                localhost, $mydomain
#mydestination = $myhostname, localhost.$mydomain, 
#                localhost, $mydomain,
#                mail.$mydomain, www.$mydomain, ftp.$mydomain
\end{lstlisting}

\$mydomainと、\$myhostnameの値が正しく設定されているかも、あらためて確認しておきましょう。

\subsection{自分が宛先ではないメールは受け取らない設定}
もうひとつ、自分宛のメールを受け取らないための設定項目があります。それは、SMTP接続を待ち受けてメールを受け取るプロセスに、定められた宛先以外のメールを受け取らない、という設定をすることです。

一見すると、この設定があれば全て問題が片付くような気がします。ですが、自分が宛先出ないメールを受け取らないということは、この前の章で試してきた、正規のユーザがメールを出せなくなる、と言うことでもあります。
そこで、他のメールサーバからノメールで、自分宛でない者を拒否しつつ、ローカルからmail(1)でメールが出せる設定がしたい、ということになりました。このように、受信するメールの条件を設定して、一致しない場合は受信を拒否するのが、smtpd\_recipient\_restrictionsというディレクティブです。

\begin{lstlisting}[basicstyle=\ttfamily\footnotesize, frame=single]
# ローカルから送信したメールと自分宛のメールは受け取る
smtpd_recipient_restrictions = permit_mynetworks ,
                               reject_unauth_destination
\end{lstlisting}

ここで、二つのあたらしいキーワード、permit\_mynetworksと、reject\_unauth\_destinationがでてきました。これらは、Postfixで規定された設定パラメータです。
Postfixの設定パラメータは、ディレクティブの引数となるときでも、\$はつけません。

判定条件を設定するディレクティブでは、その判定条件はファーストマッチです。この例では、最初にpermit\_mynetworksの条件と一致するか、判定します。この判定に成功下メールは、その次のreject\_unauth\_destinationの判定は行わず、受信します。
permit\_mynetworksに一致しなかった場合は、reject\_unauth\_destinationの条件判定が行われます。ここで、reject\_unauth\_destinationに設定された、メール受信の条件に一致したメールは受信されます。もし、どちらの条件にも一致しなかった場合は、その先の判定がないので、そのメールは受信されません。

では、これらの設定パラメータはどのような意味があるのでしょうか。

\paragraph{permit\_mynetworks}
permit\_mynetworksは、\$mynetworksで設定されたネットワークからの接続であれば、メールを受信する、という条件です。もしローカルからメールを出す必要があるときは、入れておく必要がある設定パラメータとなります。
当然、この設定パラメータを使用するときは、\$mynetworksの設定値も確認してください。

\paragraph{reject\_unauth\_destination}
許可されるべき宛先以外のメールは受け取らない、という設定パラメータです。このパラメータが設定されたとき、本書の範囲では、いかのような判定が行われます。

\begin{itemize}
  \item 受け取ったメールの宛先が\$relay]\_domainsもしくはそのサブドメインと一致している
  \item 最終的な配送先のとき、\$mydestinationか\$inet\_interfacesと一致する
\end{itemize}

\section*{}
\begin{itembox}[l]{サードパーティーリレー}
自分が宛先出ないメールを受け取ったメールサーバが、無条件にそのメールを、本来の宛先のメールサーバに転送することを、サードパーティーリレー(第三者転送)と言います。では、なぜSMTPはこのような仕組みを未だに抱えているのでしょうか。

初期のインターネット接続線は、常時接続がされていなかったり、されていたとしても回線のリソースが少なかったり、というように、今と比べるとものすごく貧弱でした。そんな中でメールを使っていくために、とりあえず近いサーバに送る、それを受け取ったサーバは、撚り近いところに転送する、というバケツリレー方式でメールを遣り取りしていたわけです。
サードパーティーリレーの機能は、そのために存在していました。

これは、インターネットのユーザがほぼ特定可能だった時代だからできた仕様ですし、世界中どこからでも、直接、宛先となるメールサーバにアクセスできる回線リソースのある現在では、もはや必要が無い機能と言えるかもしれません。

それでも、サードパーティーリレーの機能は、SMTPの中に残っています。

\end{itembox}


\section{ドメインのメールサーバをどう知らせるか}
メールの送信の説明で、送信先のドメインのメールサーバがどこか調べるには、そのドメインの権威DNSにある、MXレコードの情報を取得する、と説明しました。
では、自分がガ他のメールサーバからメールを受信するようになった時、どのサーバがドメインのメールサーバであるかを、どうやって知らせるのでしょうか。

それには、ドメインの権威DNSのMXフィールドを使います。ここでは、メールの転送を受けるメールサーバとう立場で、MXレコードを見直してみましょう。

\subsection{DNSとMXレコード}

メールサーバmail.uranohoshi.exampleを、インターネットドメインuranohoshi.exampleのメールサーバとして運用したくなったとします。どういうことかというと、たとえば、mari@uranohoshi.example宛にメールを出すと、そのメール受け取るサーバが、mai.uranohoshi.exampleにしたい、とうことです。

メールサーバの立場にでは、uranohoshi.exampleというドメインのメールサーバが、mail.uranohoshi.exampleであると判る必要がある、ということです。じつは、この情報は、メールサーバの機能だけで通知することはできません。

メールギブのメールサーバに転送するときに、DNSのMXレコードを参照しました。つまり、自分がメールを受信するときは、それを指定してやれば良いということです。
ただし、DNSは、メールとは全く別のシステムです。そそのため、ここではどのようなレコードが参照できるようにするのかを説明します。

\subsection{MXレコードに何を書くか}

メールサーバがどのホストであるかをアナウンスするためのレコードです。通常は、メールサーバのホスト名を記述します。MXレコードを参照したクライアントは、ホスト名の名前解決のために、あらためてDNSにアクセスします。そうやって取得したIPアドレスを、メールサーバへのアクセスに用います。

MXレコードは、以下のような書式で記載します。登録方法はDNSの実装について異なりますので、使っている環境に合わせてください。MXレコードは、ひとつの権威DNSに、複数登録することができます。
この例では、mail.uranohoshi.exampleと、mail2.uranohoshi.exampleの二つのメールサーバがあることを示しています。

\begin{verbatim}
IN MX 10 mail.uranohosshi.example.
   MX 20 mail2.uranohoshi.example.
\end{verbatim}

MXレコードは、ホスト名だけでなく、整数で表される優先度も記載する必要があります。この優先度で、ドメインのメールサーバにクライアントがアクセスしてくるときに、どの順番でアクセスを試みて欲しいかを伝えます。この優先度が小さい順に、クライアントはメールサーバを順番にアクセスしていくことになっています。

この例で、uranohoshi.exampleドメインのメールサーバにアクセスするクライアントは、最初に、優先度が10のmail.uranohoshi.exampleにアクセスを試みます。もし、それに失敗したときは、その次の優先度であるmail2.uranohoshi.exampleにアクセスを試みます。
そのどちらも応答がない場合は、クライアントは、uranohoshi,exampleのメールサーバが応答しない、というように判定します。

ここで、なぜMXレコードにだけ優先度があるか、という疑問が生じました。`DNSの他のレコード、たとえば、IPv4アドレスのAレコードやIPv6アドレスのAAAAレコード、ホストの別名をあらわすCNAMEレコードには、優先度というフィールドは存在しません。それは、メールサーバは冗長化の仕組みを備える必要があり、インターネットのごく初期から、その仕組みが考えられていたことによります。

\section{メールの冗長性とメールサーバの優先度}

メールサーバは、常時、インターネットに対してサービスをしていなければ鳴りません。メールを宛先に転送する、というメールの仕組みの上で、メールサーバが動いているか動いていないのか判らないというのは、その確実性を損ないます。
ですが、電子メールという仕組みがインターネットで使われ始めたごく初期は、メールサーバの信頼性、インターネットの信頼性、そのどちらもあまり高くはありませんでした。
1990年代の末ぐらいまでは、送信されたメールが届かないことも珍しくありませんでした。これは、不着が発生するという意味でも、郵便のメタファーであったわけです。

このような信頼性が高くない状況で、確実にメールを届けるにはどうするか。メールサーバは、転送できなかったメールを、受け取ってから5日の間、メールキューに保持します。そして、5日の間再送を試み、それでも失敗した、転送できなかったメールを破棄します。これは、一定期間再送を試みることであり、メールサーバに繋がるのを待つという形の冗長性の取り方です。

もうひとつのやり方で、メールサーバを複数おいて、冗長性を取るのいうやりかたも考えられます。ですが、メールサーバの機能を持つホストを複数置くだけというやり方を採ることはできません。なぜなら、メールが最終的に格納されるメールボックスは、どこのメールサーバの上にあるか、特定できなければならないからです。同じ機能をもつメールサーバが複数合っても、どこのメールサーバのメールボックスにメールが届いたか特定できなければ、ユーザは届いたメールを最終的に受け取ることができなくなります。

そこで考えられたのが、メールサーバに優先度を着けて、クライアントには優先度順にアクセスを試みてもらうという方法です。

\subsection{プライマリとセカンダリ}

メールサーバでメールボックスを持つものは、ドメインで一つだけにします。これでどうやって冗長性をかくほするかというと、同じドメインのメールサーバを、メールボックスを持つメールサーバいかいにもいくつか置きます。そして、メールボックスを持つメールサーバ以外のメールサーバは、メールボックスを持つメールサーバに、届いたメールを全て転送する設定とします。

通常の運用では、メールボックスを持つメールサーバは、そのドメインで一番高い優先度を持たせます。優先度の数値では、最も小さい値を設定します。この、ドメインのメールサーバの中で最も優先度が高い(優先度の数値が小さい)メールサーバを、プライマリメールサーバ、あるいは、メールサーバの文脈では、単にプライマリと呼びます。
そして、プライマリ以外のメールサーバを、セカンダリメールサーバ、あるいは単にセカンダリ、と呼びます。
そして、DNSのMXレコードに、プライマリメールサーバとセカンダリメールサーバの情報を記載します。

先程のMXレコードの例をもう一度見てみましょう。

\begin{verbatim}
IN MX 10 mail.uranohosshi.example.
   MX 20 mail2.uranohoshi.example.
\end{verbatim}

このMXレコードの例では、優先度が10で、最も小さい、つまり最も優先度が高いプライマリメールサーバは、mail.uranohoshi,exampleです。そして、優先度の数値が20で、それより優先度が低いmail2.uranohoshi,exampleが、セカンダリメールサーバとなります。

この方法で冗長性ととった場合、プライマリメールサーバのサービスが停止していたら、どのようにメールが届くのでしょうか。プライマリメールサーバが停止していたとすると、クライアントは、MXレコードの情報に従って、セカンダリメールサーバにメールを転送します。
セカンダリメールサーバは、プライマリメールサーバに受信したメールを転送しようとします。もし、プライマリメールサーバが応答しない場合は、セカンダリメールサーバはメールキューにメールを置き、5日間の間、プライマリメールサーバへの手の素を試みます。

\subsection{Postfixのセカンダリメールサーバ設定}
Pottfixでは、セカンダリメールサーバはどのような設定をすればよいのでしょうか。セカンダリメールサーバでは、\$relayhostの値として、プライマリメールサーバを設定します。、また、\$relay\_domainsの値として、プライマリメールサーバに転送するメールのドメイン、つまり自分のドメイン名を書きます。

セカンダリメールサーバmail2.uranohoshi.exampleが、インターネットドメインuranohoshi.example宛のメールを受信したら、全てプライマリメールサーバである、mail.uranohoshi.exampleに転送します。
この動作似必要な、main.cf(5)への設定は以下のようになります。

\begin{lstlisting}[basicstyle=\ttfamily\footnotesize, frame=single]
# uranohoshi.exampleを受信して転送するドメイン
mydomain = uranohoshi.example
relay_domains = $mydomain

# 全てのメールをmail.uranohoshi.example二転送
relayhosts - [mail.uranohoshi.example]
\end{lstlisting}

\subsection{現在のプライマリメールサーバの考え方}

ここまでで、ドメインのプライマリメースサーバは、そのドメインのメールボックスを持つメールサーバであるという説明をしました。実はこれは古い概念で、現在は、外部からアクセスさせたいメールサーバがプライマリメールサーバとなります。セカンダリメールサーバについても、その優先度の違いだけで、外部からアクセスさせたいメールサーバで、優先度が次点以降のもの、という位置づけとなります。

たとえば、セキュリティチェックを専門に実行するメールサーバを用意したとしましょう。ドメイン宛のメールは、全てここに転送させるよう、DNSのMXレコードを設定します。
このとき、プライマリメールサーバは、メールボックスを持っていません。
そして、セキュリティチェックが終わったメールのみを、メールボックスを持つメールサーバに転送する、という運用を行っていれば、メールボックスをもつメールサーバは、プライマリでもセカンダリでもる必要がありません。


\section{不正なホストからのメール転送拒否}

メールを受信するときは、不正なホストからのメール転送をなるべくブロックしておきたいところです。メール送信のところで、S25RやGlaylisting、SPF、DKIMといった、メールサーバの正当性を証明する方法を紹介しました。Postfixではこれらをどのように利用すれば良いのでしょうか。

実のところ、Postfixには、受信時にこれらをチェックする機能はありません。これらのチェック機能は、アクセスポリシー委譲やmilter呼び出しという機能で、外部プロセスに判定してもらい、結果をもらって判断する形を取ります。そのため、本書の範囲を超えるため、ここでは紹介に留めます。

\subsection{アクセスポリシー委譲}
アクセスポリシー委譲は、メールに関する情報を外部プロセスに送ります。メールそのものではなく、送信してきたホスト、メールの送信元や宛先、といった情報のみを送ります。そのため、メールに関する情報のみで判別を行うことができる、グレイリスティングなどがアクセスポリシー委譲のプロセスとして実装されています。

外部プロセスはその情報を受け取って、そのメールについて判定情報をかえします。Postfix八草の情報をもとに、そのメールを受け取るかどうかを判断します。

\subsection{milter}
milterとは、mail filterの略で、もともとは、sendmailに、メールフィルタ機能を追加するための規格でした。Postfixは、sendmailのmilterと互換性があり、そのまま使うことができます。

Postfixからmilterに、SMTPでメールそのものを転送します。また、Postfixへは、判定結果やメールの書き換え指示が返されます。メールそのものが戻されるわけではありません。
メールそのものを渡すことから、milterはスパムやウィルスをチェックする機能や、メールのヘッダ情報を参照するDKIMなどが実装されています。