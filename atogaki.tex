\chapter{あとがき}

\section*{メイ・カートミル(仮名)とマージ・ニコルス(仮名)のお台場での会話}

\begin{quotation}
\noindent
{\bf メイ}「リア充がこんなにたくさんいるなんてわけがわからないよ、ですね」\\
{\bf マージ}「クリスマスイブのお台場を正直嘗めていたわ」 \\
{\bf メイ}「マージさん、リア充爆発しろ!です」 \\
{\bf マージ}「…自爆テロ乙」 \\
\end{quotation}

\section*{あとがき}

お兄ちゃん、中折れコピー本の限界に挑戦するってこういうことだね。

というわけで、『同人版TCP/IP入門別冊 SSLのはなし』をほぼ全面改稿した、同人版TLS/SSL入門を出すことができました。今回は、コピー本であるため、2分冊しています。

SSLのはなしは、2011年に出してから、2014年まで都合7回再版しています。このあたりは、コピー本の気楽さということでしょうか。ですが、流石に内容の古さが気になり、しばらく封印していたものを、現在の情報を加えて、新しい本として出すこととなりました。
それにあたっては、今回はわたしがリライトを担当、主宰がフォローという形を取っています。

前回の後書きを見ると、日本語で読めるTLSに着いての参考文献の少なさについて書いていましたが、現在でもその状況はあまり変わっていないようです。TLSの基板となる技術については、ググれば、個々の説明を見つけることはできます。でも、TLSがどのように成り立っているかについてを、纏めることができたかなという自負はあります。

さいごになりますが、今回表紙イラストを担当してくれたゆうちゃんさん、上下2分冊コピー本という無茶にもかかわらずかわいい表紙を提供してくれて、ありがとうございました。、

\begin{flushright}
2018年4月21日 \\
インフラエンジニアの毒舌な妹(@infra\_imouto)
\end{flushright}

\section*{後書き}

まずは、表紙イラスト担当のゆうちゃんに謝辞を。一気に華やかな内容になりました。

本書は、以前の版を私が書きましたが、今回はインフラエンジニアの毒舌な妹にリライトを任せました。それなりに読みやすいと思っていただければ、共著者としては幸いです。

SSLの話を出した7年前は、何でもかんでもSSLにすれば良いものではない、というのがそのテーゼでした。ですが、時代も変わり、SSLは、暗号化と同時に、認証ためにも必要になっています。だからこそ、オレオレ証明を何故使ってはいけないのか、それについての記述を増やしたりもしました。

TLSは、アプリケーション感通信の要素技術です。ですが、理解をしようとすると、整数論と暗号化、PKI、TCP/IPという具合に、いろいろと前提となる知識が必要になります。本書では数学的な議論はしていませんが、それでも、TLSについて、より明確にイメージを持っていただければ、著者としてこれ以上の喜びはありません。

\begin{flushright}
2019年4月21日 \\
ありす ゆう
\end{flushright}


%\newpage

% 1ページブランクを入れる

%\thispagestyle{empty}
%\mbox{}
%\newpage
%\clearpage


\thispagestyle{empty}

\vspace*{\fill}
\begin{tabular}{ll} \toprule
筆者 & インフラエンジニアの毒舌な妹 ありす ゆう \\
表紙イラスト & ゆうちゃん(コース英知) \\
発行 & AliceSystem \\
連絡先 & aliceyou@alicesystem.com \\
URL & http://aliceyou.air-nifty.com/onesan/ \\
初版発行日 & 2018年4月22日 \\
初版二刷発酵日 & 2018年7月2日 \\
印刷所 & Kinko's \\ \bottomrule
\end{tabular}