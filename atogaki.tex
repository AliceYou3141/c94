\chapter{あとがき}

\section*{メイ・カートミル(仮名)とマージ・ニコルス(仮名)の西伊豆スカイラインでの会話}

\begin{quotation}
\noindent
{\bf メイ}「マージさん、ここを免許取って半年で夜中に走るんですよ」\\
{\bf マージ}「無茶しやがって」 \\
{\bf メイ}「マージさん、植生限界ですよ」 \\
{\bf マージ}「稜線を縦走してるような道路だね、ここ」 \\
\end{quotation}

\section*{あとがき}

お兄ちゃん、分量的に無理なものは無理。

前書きで結構な危機感をもって書いた一冊です。でも、メールというサービスと、Postfixの基礎で一冊としました。メールというエコシステムを作るには、ユーザアカウントデータベースをどうするか、ユーザが送信するメールをどう受け付けるか、そしてMRAをどうするか、そしてDNSの設定、そう言った部分を知る必要があります。
本気で書くとユーザアカウントデータベースだけで一冊必要となりますので、ページ数というか印刷費の都合で……

それでも、失われつつあったPostfixの基礎的な設定を纏めることができ、Postfixと15年ぐらい付き合ってきた主宰の知識の集大成の、最初の一冊になったのかなと思います。そして、多分、日本語で読める最後の、いえ、一番新しいPostfixの本です。
これからPostfixを使っていくことになる皆さんの、知識の一部となれるなら、これほどうれしいことはありません。
では、Postfixについては、また続刊でお会いしましょう。

最後になりますが、今回も表紙イラストを担当してくれたゆうちゃんさん、メールに鳩というアバウトなオーダーでかわいい表紙を提供してくれて、ありがとうございました。、

\begin{flushright}
2018年8月10日 \\
インフラエンジニアの毒舌な妹(@infra\_imouto)
\end{flushright}

\section*{後書き}

まずは、表紙イラスト担当のゆうちゃんに謝辞を。本書も、一気に華やかな内容になりました。

今回も、Postfixの本ということで、インフラエンジニアの毒舌な妹にメイン部分を任せ、私は検証環境や掲載用サンプルの作成などを行っています。
本書は、AliceSystemのPostfix本としては2冊目になります。そして、Postfixの入門書としてははjめてです。なにせ、前に出したPostfixプラス+は、いきなりmaster.cf(5)の設定の仕方から始まって、メールルーティングやフィルタリングの技芸のようなことだけをまとめた一冊でした。

やはり、Postfixだけでなく、メールサーバの設定情報が書籍としては失われてしまった今の状況に、これはどうにかしないとと思ったところから、この企画がスタートしています。なにより、基礎的な部分をみんなが判っている上級にしないと、Postfixプラス+の方は第四版を出せないし、周到に仕込んだゲームのネタで笑ってもらうこともできなくなります。

冗談はともかくとして、DevsOpsの流れで、メールサーバも構築しなければならない事例が出てきているはずなのですが、そのための情報が散逸している今の状況は、あまり好ましいとは思えません。このあたりは、インフラエンジニアの毒舌な妹(の中の人)と、意見をおなじくするところです。
かといって、メールというサービスを提供するとしたら、関連知識の範囲が大きくなりすぎます。広げた風呂敷をたためな無かったと言われればそれまでせすが、まずはPostfixの基礎的な扱いについて纏めたということで、よしとしています。

なお、例示用尾ドメインについては、私の趣味です。本文コラムでも書きましたが、どうしてもexample.comとexample.orgでは見た目的に紛らわしいので、例示用TLDの.exampleを使用しました。ほとんどuranohoshi.exampleしか使っていませんが、初期の構想では、otonoki.example宛にメールを出すためにAqoursがヨハネちゃんを中心としてメールサーバ構築に奔走する、という筋が気にするアイディア名残です。流石にこの路線はビジュアルがないと、とうことでボツになりましたが。

では、また次の本でお目にかかるのを楽しみにしつつ、今回は筆を置。

\begin{flushright}
2019年8月10日 \\
ありす ゆう
\end{flushright}


%\newpage

% 1ページブランクを入れる

\thispagestyle{empty}
\mbox{}
\newpage
\clearpage


\thispagestyle{empty}

\vspace*{\fill}
\begin{tabular}{ll} \toprule
筆者 & インフラエンジニアの毒舌な妹 ありす ゆう \\
表紙イラスト & ゆうちゃん(コース英知) \\
発行 & AliceSystem \\
連絡先 & aliceyou@alicesystem.com \\
URL & http://aliceyou.air-nifty.com/onesan/ \\
初版発行日 & 2018年8月10日 \\
印刷所 & ...... \\ \bottomrule
\end{tabular}