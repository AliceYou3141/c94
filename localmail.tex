\chapter{MTAその1: Postfixの基本操作}

\section{Postfixの設定ファイル}
Postfixの設定ファイルには、インストール時に配置されるものと、sendmail互換のために使用されるものと二種類があります。設定ファイルにはいくつかありますが、本書で取り上げるものの概要を説明します。
内容については、実際にそのファイルを書き換える必要が出たときに、あらためて説明します。

\subsection{Postfix由来の設定ファイル}

Postfixの設定ファイルは、通常は/etc/postfixの下に置かれます。これはPostfixが標準のMTAとなっているOSの多くと、RPMでパッケージ管理をするLinuxのディストリビューションの場所です。

FreeBSDでPortsやPackagesでPostfixを導入した場合は、設定ファイルは/usr/local/etc/postfixの下に置かれます。

導入したばかりの状態では、/etc/postfixもしくは、/usr/local/etc/postfixには、およそ以下のファイルが含まれています。これは、OS毎に提供されるパッケージによっては異なることがあります。この章では、main.cfの必要な部分を書き換えます。

\begin{lstlisting}[basicstyle=\ttfamily\footnotesize, frame=single]
+-main.cf
+-master.cf
+-transport
\end{lstlisting}

\paragraph{main.cf}
main.cf(5)は、Postfixの動作を設定するファイルです。Postfixの基本的な設定は、main.cfに記述されます。

\paragraph{master.cf}
master.cf(5)は、Postfixのプロセスを制御する設定ファイルです。Postfix
のプロセスを管理する、master(8)が参照します。

\paragraph{tranxport}
transport(5)は、メールの転送ルールを記述するファイルです。特定の宛先のメールを、決められたメールサーバに転送したいというようなときに使用します。
transport(5)は通常、その内容をpostmap(1)コマンドでハッシュします。Postfixは、postmap(1)が生成した、transport.dbというファイルを参照します。


\subsection{sendmail互換のための設定ファイル}
Postfixはもともと、sendmailというレガシーなMTAと、コマンドレベルで互換性を保ちつつ、近代的な実装をするという目的で開発されました。そのため、sendmailと同じ形式の設定ファイルを使うようになって言い留ものがあります

\paragraph{/etc/alias}
メールのエイリアス(別名)を設定するファイルです。互換性のため、FreeBSDでも/etc/aliasとして置かれます。

エイリアスは、あるメールアドレスの別名になるメールアドレスを設定して、エイリアスのアドレス宛に届いたメールは、そのエイリアスを別名として持つメールアドレスに転送されます。よくある使い方として、ホストの管理者のメールアドレスであるroot@localhostを、実際に管理を行っているユーザのメールアドレスのエイリアスとして設定します。そうすることで、root宛のメールが、管理者ユーザの持つメールアドレスに転送されます。

\paragraph{/etc/mail/mailer.conf}
FreeBSDなど、sendmailが標準インストールされるOSで、semdmail互換のために置かれるファイルです。CentOSなど、MTAが標準インストールされない環境には存在しません。

FreeBSDにPostfixを導入すると、sendmail由来のnewaliases(1)、mailq(1)、sendmail(1)などのコマンドは、すべてmailwrapper(1)というラッパーへのリンクになります。

mailwrapper(1)は、/etc/mail/mailer.confの内容を見て、sendmail(1)とmailq(1)とnewaliases(1)について、指定されたパスにあるバイナリを実行します。


\section{Postfixの基本操作}

\subsection{Postfixを起動する}
まずはPostfixを起動しましょう。Postfixが起動しているかどうかは、masterというプロセスが動いている華道家で確認することができます。

\begin{lstlisting}[basicstyle=\ttfamily\footnotesize, frame=single]
root # ps ax | grep master
  675  -  Ss    0:00.06 /usr/local/libexec/postfix/master -w
\end{lstlisting}

masterというプロセスが動いていない場合は、Postfixがは起動していません。以下のようにPostfixを起動しましょう。

\begin{lstlisting}[basicstyle=\ttfamily\footnotesize, frame=single]
# postfixコマンドを使う場合
postfix start

# serviceコマンドがある環境
# CentOS6やFreeBSDなど
service postfix start

# systemdで管理されている場合(CentOS7はど)
systemctl start postfix.service
\end{lstlisting}

\subsection{OS起動時にPostfixが起動するように設定する}

OS起動の時点でPostfixが立ち上がるようにするには、このように設定します。

CentOSの場合は、以下のようにコマンドで設定します。

\begin{lstlisting}[basicstyle=\ttfamily\footnotesize, frame=single]
# CentOS6の場合
chkconfig postfix on

# systemdで管理されている場合(CentOS7はど)
systemctl enable postfix.service
\end{lstlisting}

FreeBSDの場合は、/etc/rc.confに以下のように記入します。

\begin{lstlisting}[basicstyle=\ttfamily\footnotesize, frame=single]
postfix_enable="YES"
\end{lstlisting}

\subsection{Postfixを停止する}
Postfixを停止するときは、コマンドラインから、以下のように操作します。いずれも、Postfixの全ての動作を停止するコマンドです。

\begin{lstlisting}[basicstyle=\ttfamily\footnotesize, frame=single]
# postfixコマンドを使う場合
postfix stop

# serviceコマンドがある環境
# CentOS6やFreeBSDなど
service postfix stop

# systemdで管理されている場合(CentOS7はど)
systemctl stop postfix.service
\end{lstlisting}



\subsection{設定ファイルの再読込}
Postfixの設定を変更した場合、その都度、設定ファイルを再読込させる必要があります。設定の再読込を行うときは、管理者のアカウントで、以下のように実行します。

\begin{lstlisting}[basicstyle=\ttfamily\footnotesize, frame=single]
# postfixコマンドを使う場合
postfix reload

# serviceコマンドがある環境
# CentOS6やFreeBSDなど
service postfix reload

# systemdで管理されている場合(CentOS7はど)
systemctl reload postfix.service
\end{lstlisting}

\subsection{Postfixの再起動}
設定変更の内容によっては、Postfixの再起動が必要となる場合があります。Postfix事態を再起動するときは、コマンドラインから、以下のように操作します。

\begin{lstlisting}[basicstyle=\ttfamily\footnotesize, frame=single]
# postfixコマンドを使う場合
postfix restart

# serviceコマンドがある環境
# CentOS6やFreeBSDなど
service postfix restart

# systemdで管理されている場合(CentOS7はど)
systemctl restart postfix.service
\end{lstlisting}

\section{Postfixのログ}

\section{自分宛のメールを受け取るために}

まず、Postfixが動いているホストにログインして、いまログインしているユーザ宛にメールを出し、受け取って見ましょう。そのためには、Postfix二、以下のような設定を行う必要があります。

メールサーバの名前
このメールサーバで受け取ってメールサーバに入れる宛先ドメイン
どこのネットワークからPostfixへの接続を許可するか

\subsection{どこからの接続をゆるすか}
