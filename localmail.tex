\chapter{MTAその1: Postfixの基本操作と最初のメール}

\section{Postfixの基本操作}

\subsection{Postfixを起動する}
まずはPostfixを起動しましょう。Postfixが起動しているかどうかは、masterというプロセスが動いている華道家で確認することができます。

\begin{lstlisting}[basicstyle=\ttfamily\footnotesize, frame=single]
root # ps ax | grep master
  675  -  Ss    0:00.06 /usr/local/libexec/postfix/master -w
\end{lstlisting}

masterというプロセスが動いていない場合は、Postfixがは起動していません。以下のようにPostfixを起動しましょう。

\begin{lstlisting}[basicstyle=\ttfamily\footnotesize, frame=single]
# postfixコマンドを使う場合
postfix start

# serviceコマンドがある環境
# CentOS6やFreeBSDなど
service postfix start

# systemdで管理されている場合(CentOS7はど)
systemctl start postfix.service
\end{lstlisting}

\subsection{OS起動時にPostfixが起動するように設定する}

OS起動の時点でPostfixが立ち上がるようにするには、このように設定します。

CentOSの場合は、以下のようにコマンドで設定します。

\begin{lstlisting}[basicstyle=\ttfamily\footnotesize, frame=single]
# CentOS6の場合
chkconfig postfix on

# systemdで管理されている場合(CentOS7はど)
systemctl enable postfix.service
\end{lstlisting}

FreeBSDの場合は、/etc/rc.confに以下のように記入します。

\begin{lstlisting}[basicstyle=\ttfamily\footnotesize, frame=single]
postfix_enable="YES"
\end{lstlisting}

\subsection{Postfixを停止する}
Postfixを停止するときは、コマンドラインから、以下のように操作します。いずれも、Postfixの全ての動作を停止するコマンドです。

\begin{lstlisting}[basicstyle=\ttfamily\footnotesize, frame=single]
# postfixコマンドを使う場合
postfix stop

# serviceコマンドがある環境
# CentOS6やFreeBSDなど
service postfix stop

# systemdで管理されている場合(CentOS7はど)
systemctl stop postfix.service
\end{lstlisting}

\subsection{設定ファイルの場所}

Postfixの設定ファイルは、通常は/etc/postfixの下に置かれます。これはPostfixが標準のMTAとなっているOSの多くと、RPMでパッケージ管理をするLinuxのディストリビューションの場所です。

FreeBSDでPortsやPackagesでPostfixを導入した場合は、設定ファイルは/usr/local/etc/postfixの下に置かれます。

導入したばかりの状態では、このディレクトリには、およそ以下のファイルが含まれています。この章では、main.cfの必要な部分を書き換えます。

\begin{lstlisting}[basicstyle=\ttfamily\footnotesize, frame=single]
+-main.cf
+-master.cf
+-transport
\end{lstlisting}

\subsection{設定ファイルの再読込}
Postfixの設定を変更した場合、その都度、設定ファイルを再読込させる必要があります。設定の再読込を行うときは、管理者のアカウントで、以下のように実行します。

\begin{lstlisting}[basicstyle=\ttfamily\footnotesize, frame=single]
# postfixコマンドを使う場合
postfix reload

# serviceコマンドがある環境
# CentOS6やFreeBSDなど
service postfix reload

# systemdで管理されている場合(CentOS7はど)
systemctl reload postfix.service
\end{lstlisting}

\subsection{Postfixの再起動}
設定変更の内容によっては、Postfixの再起動が必要となる場合があります。Postfix事態を再起動するときは、コマンドラインから、以下のように操作します。

\begin{lstlisting}[basicstyle=\ttfamily\footnotesize, frame=single]
# postfixコマンドを使う場合
postfix restart

# serviceコマンドがある環境
# CentOS6やFreeBSDなど
service postfix restart

# systemdで管理されている場合(CentOS7はど)
systemctl restart postfix.service
\end{lstlisting}

\section{自分宛のメールを受け取るために}

まず、Postfixが動いているホストにログインして、いまログインしているユーザ宛にメールを出し、受け取って見ましょう。そのためには、Postfix二、以下のような設定を行う必要があります。

メールサーバの名前
このメールサーバで受け取ってメールサーバに入れる宛先ドメイン
どこのネットワークからPostfixへの接続を許可するか

\subsection{どこからの接続をゆるすか}
