\chapter{ほかのメールサーバにメールを転送する}

次は、先程sって伊下メールサーバ以外の、ネットワーク接続されたほかのメールサーバにメールを転送してみましょう。実は、前章までの設定で、帰納的には外にメールを出すのに必要な、最低限の設定はできています。ですが、現在のインターネットでは、そのままではメールが届かなかったり、そのままメールを送ってしまうと問題になったりします。

この章では、メールを外に届けるとはどういうことか、それを見ていくことにしましょう。

\section{名前解決と権威DNS}

これから先の話をするときに、最低限のDNSに関する知識が必要となります。メールのサービスはDNSを前提としているので、正しい運用には、DNSの知識も不可欠です。

本書はPostfixの本なので深くは立ち入りませんが、今後の説明に必要になる部分について、DNS二巻する説明をします。

\subsection{名前解決}
インターネットのホストにアクセスするとき、アクセス先を表すのに、通常はwww.uranohoshi.exampleやmail.utx.exampleのように、干すトメイと呼ばれる名前を使用します。192.168.1.1や2001:db8::1/64のような、IPアドレスでもアクセスできますが、人間の負担が大きいので、通常は使用しません。

ですが、実質的にスタンダードであるプロセス間通信の手段であるsocker(2)は、接続先を表現するのに、IPアドレスを必要とします。そのため、socket(2)を利用するソフトウェア側で、何らかの手段でホスト名とIPアドレスをマッピングする手段を用意する必要があります。
何らかの手段で、ホスト名とIPアドレスのマッピングができたとき、それを名前解決ができた、表現します。名前解決の方法には、DNSに情報を問い合わせる、ローカルのマッピングテーブルを参照する、などがあります。

\subsection{権威DNS}
あるインターネットドメインの権威DNSとは、そのドメインについて全てのホスト名の情報を持っていて、外部からそのドメインのホストに関する問い合わせに回答する権限を持つものを指します。

たとえば、uranohoshi.exampleドメインの権威DNSは、FQDNにuranohoshi.exampleというインターネットドメインを持つホストについて、名前解決の問い合わせに応える機能と情報を持っています。\footnote{サブドメインを別のDNSに管理してもらう委譲の概念もありますが、本書では深く立ち入りません。DNS関連の資料を参照してください。}
また、権威DNSは、そのドメインの権威DNSの名前や、メールサーバの名前などの情報も持っています。

\subsection{逆引き解決}
名前解決は、ホスト名と、対応するIPアドレスのマッピングを行うことでした。その逆に、IPアドレスからホスト名の方向のマッピングをすることを、逆引き解決とよびます。

逆引き解決は、多くの場合、それ用のDNSによって提供されます。また、この逆引き用DNSは、ドメインの権威DNSとは、関係ないホストで運用されることも多いです。

\section{ほかのメールサーバを知る}

他のメールサーバの存在は、どのように知れ知れば良いのでしょうか。Postfixをはじめとする`MTAは、自分自身が管理するメールアドレスが宛先でないメールは、全て他のサーバに転送しようとします。このとき、どこのサーバに転送すれば良いかを知る必要があります。

そのためには、メールサーバは、DNSによる名前解決ができなければなりません。では、メールサーバではどのような情報が得られるのでしょうか。

\subsection{他のメールサーバ接続する}
メールアドレスchika@mail.uranohosi.exampleから、honoka@mail.otonoki.exampleにメールを出そうとしたとします。この場合は、メールボックスが存在するホスト名がメールアドレスとして書かれているので、ホスト名の名前解決ができれば、メールを送ることが可能です。
メールサーバが他のメールサーバに接続するとき、INETドメインで25/TCPのソケットを開かなければ鳴りません。そのため、何らかの手段で、ホスト名とIPアドレスを対応づける方法が必要です。そして、その多くは、DNSで、AレコードもしくはAAAAレコードを参照し、その結果情報を利用して、メールサーバに接続します。

\subsection{他のドメインのメールサーバを知るには}

メールの送信先アドレスとして、honoka@otonoki.exampleと、メールアドレスがドメイン名のみで記載されている場合は、どうやって、otonoki.exampleのメールサーバの名前やIPアドレスを知れば良いのでしょうか。

これも、DNSの情報を参照します。メールサーバがドメインに対してメールを転送するとき、メールサーバはDNSに、そのドメインのMXというレコードの情報を問い合わせます。このMXは、Mail eXchanger、つまり、メールサーバという意味をもったレコード名です。
あるインターネットドメインの権威DNSは、そのドメインのメールサーバの情報を、MXレコードとして回答します。そのMXレコードの例をしめすと、のようになります。

\begin{verbatim}
otonoki.example.  86400  IN  MX 10 mail.otonoki.example
                  96400  IN  MX 20 mail2.otonoki.example
\end{verbatim}

MXレコードの回答を見ると、otonoki.exampleドメインのメールサーバは、mail.otonoki.exampleとmail2.otonoki.exampleの二つあることがわかります。では、このふたつのレコードに記載された数字は何なのでしょうか。

最初の86400は、TTL(Time To Live)と言われるフィールドです。この情報をキャッシュしてもかまわない秒数を表します。この例だと、86400秒、つまり有効時間は1日です。TTLは強制で無く、問い合わせる側に向けた指標です。そのため、TTL時間いっぱい前回の問い合わせをキャッシュして使う、という運用はあまりありありません。

次に現れている、10や20という数字は、メールサーバの優先度です。どのドメインに複数のMXがあるときは、優先度の数字がが小さいものから、メールの転送を試みることになっています。この例では、最初に優先度10のmail.otonoki.exampleにSMTPでの接続を試みます。それが失敗した場合、その次の優先度である、mail2.otonoki.exampleに接続を試みます。

DNSのMXレコードについては、外部からメールを受け取るための設定の説明を行う際に、再度取り上げます。

\section*{}
\begin{itembox}[l]{DNSラウンドロビン}
もし、複数のMXレコードで優先度が同じであったら、どのメールサーバが利用されるのでしょうか。同じ優先度のメールサーバが複数あるときは、そのうちから1台、確率的に選び出します。そのため、同じ優先度のメールサーバが複数あるときは、同じ確率でどれかが選ばれる、というのが正解です。

また、ひとつのホスト名に対して、複数のAレコードや、複数のAAAAレコードを設定することもあります。この場合も、Aレコード、もしくはAAAAレコードのどれかが、確率的に選択されます。

このようなDNS情報の設定の仕方を、DNSラウンドロビンと読んでいます。複数あるレコードのどれかが選ばれる状況を、駒鳥(ロビン)が素に戻るときにぐるぐると同じ所を廻って、素の場所を特定されるのを避けている行動になぞらえたものです。
\end{itembox}


\section{OPB25とS25R}
前章で行った設定と、メールサーバが名前解決が行えるだけの設定をすれば、それでメールサーバに、外部にメールを送る機能が備わったことになります。ですが、現在のインターネットでは、それだけでは、他のメールサーバにメールを転送することができません。それは、OPB25と、S25Rというセキュリティ手法Jのためです。

\subsection{OPB25}
OPB25とは、Outbound Port Blocking 25の略で、外部25番ポートへの接続ブロッキング、という意味になります。OP25や、OP25Bとしている資料もあります。

不正なメール送信を防ぐ目的で、インターネットプロバイダが、そのプロバイダがサービスとして提供しているメールサーバ以外、つまり、そのプロバイダのネットワークの外にあるメールサーバの25/TCPへのアクセスをブロックする、というセキュリティ手法です。この設定がされているインターネットプロバイダを利用している場合、インターネットに接続されたメールサーバの25/TCP二アクセスできません。つまり、SMTPで接続ができません。

O{B25は、インターネット接続を利用する個人ユーザに向けて設定されることが多い制限です。

\subsection{S25R}

メールサーバに接続してきたホストのIPアドレスから逆引きをして、そのIPアドレスに対応するホスト名がメールサーバっぽいかどうかで接続を許可するか拒否するか判断する手法です。S25Rとは、Selective SMTP(25) Restrictionの略で、選択的SMTP制限、という用な意味になります。

インターネットプロバイダを使用する個人ユーザに割り当てられるIPアドレスに対応する名前は、一見して、その用途であると判るような名前を付けられています。
このような名前が付け羅得ているホストからのアクセスは、正規のメールサーバでないとみなして、接続を拒否します。
また、逆引きで得たドメイン名の情報が、送信されたメールの発信元のドメインと一致しなかったり、逆引き解決ができない場合も、同様に接続拒否をします。

\subsection{S25R+クレイリスティング}
セキュリティの都合や、複数ユーザ相乗り型のレンタルサーバであるなどの理由で、逆引きができないホストが多く、S25Rをそのまま使うのは、拒否する必要が無いホストからの接続を拒否してしまう、フォルスポジティブが発生します。

そのフォルスポジティブを減らすために、S25Rを、グレイリスティングのトリガーにするという方法があります。このグレイリスティングは、正規のメールサーバかどうか疑わしいホストからの接続を、初回接続から5分間拒否するという手法です。
メールサーバとして動作しているホストであれば、規約に従って、接続に失敗したホストに、5分以上後に再接続を試みます。そうやってメールを送信してきたホストを、ホワイトリストに入れ,一定期間SMTP接続を認める、という手法です。

その一方、不正なメールを送信しようとするホストは、多くの場合、5分後に再送信をしません。そのため、グレイリスティングによって、ほとんどの不正なホストからのSMTP接続をブロックすることができます。




\section{全てのメールを特定のメールサーバに送る}
メールサーバに届いたメールで、そのメールサーバ自身が宛先でないメール全てを、特定のメールサーバに送ることができます。たとえば、LANの中でインターネットに接続されたメールサーバが1台だけあり、外にメールを出すには、そのサーバに一度メールを転送しなければいけない、というような場合に、この設定を行います。

注意として、この設定で転送先とするサーバの管理者から、全てのメールを転送することを許可されている状態で行ってください。そうでないなら、メールを送りつける攻撃になってしまいます。

全てのメールを転送素には、relayhostディレクティブに、転送先のメールサーバを設定します。デフォルト状態で、\$relayhostは値が設定されていません。

main.cf(5)でコメントアウトされている設定例として、このようなものが挙げられています。

\begin{lstlisting}[basicstyle=\ttfamily\footnotesize, frame=single]
#relayhost = $mydomain
#relayhost = [gateway.my.domain]
#relayhost = [mailserver.isp.tld]
#relayhost = uucphost
#relayhost = [an.ip.add.ress]
\end{lstlisting}

Postfixでは、メールを転送する先を、角括弧の有無で区別します。それは、DNSでMXレコードを検索するかどうかの違いです。

\subsection{MX検索を行う場合}
\$relayhostの値が角括弧無しのとき、Postfixは、その値を転送先のドメイン名であると見なします。そして、DNSから転送先となるドメインのMXを検索し、MXレコードに記載されたメールサーバの転送を試みます。

以下のようき記載した場合、utx.exampleのMXを検索し、utx.exampleドメインのメールサーバに全てのメールを転送します。

\begin{lstlisting}[basicstyle=\ttfamily\footnotesize, frame=single]
relayhost = utx.example
\end{lstlisting}

\subsection{MX検索を行わない場合}
角括弧有りの場合は、MX検索は行いません。角括弧の中に書かれた内容がIPアドレスかホスト名歌を判別します。そして、ホスト名の場合は、名前解決を行います。また、角括弧の後にコロンを付けて、ポート番号を指定することもできます。

以下の設定例では、mail.otonoki.exampleのポート2525に、全てのメールを転送する設定です。

\begin{lstlisting}[basicstyle=\ttfamily\footnotesize, frame=single]
relayhost = [mail.otonoki.examle]:2525
\end{lstlisting}



\section{インターネット接続されたメールサーバにメールを送る}

では、インターネット接続されたメールサーバに、宛先の応じてメールを転送するにはどのようにすれば良いでしょうか。それは、OPB25やS25Rの規制を受けない環境でメールサーバを運用することと、他のメールサーバから、SMTP接続を受けて良いサーバとみなされる設定にすることです。

\subsection{メールサーバをどこに置くのか}

\subsection{メールサーバの信用}