\begin{thebibliography}{99}
\item
	Rescorla,Eric
	斎藤孝道 鬼頭利之 古森貞 監訳
	2003
	『マスタリングTCP/IP SSL/TLS編』
	東京:オーム社
\item
	Risti\'c,I
	斎藤孝道 監訳
	2017
	『プロフェッショナルSSL/TLS』
	東京:ラムダノート株式会社
\item
	Nash,Andrew Duane,William Joseph,Celia Brink,Derek
	株式会社スリーエーシステムズ 訳
	2002
	『PKI eセキュリティの実装と管理』
	東京:翔泳社
\item
	笠野英松 監修
	マルチメディア通信研究会 編
	1998
	『ポイント図解式 インターネットRFC事典』
	東京:アスキー出版局
\item
	小俣光之 種田元樹 著
	2011
	『Linuxネットワークプログラミングバイブル』
	東京:秀和システム
\item
	Viega,John Messier,Matt Chandra,Pravir
	斎藤孝道 訳
	2004
	『OpenSSL 暗号・PKI・SSL/TLSライブラリの詳細』
	東京:オーム社
\item
	Leffer,S.J. McKusick,K.M. Karels,M.J. Quarterman,J.S.
	中村明 相田仁 計宇生 小池汎平 共訳
	1991
	『UNIX 4.3BSDの設計と実装』
	東京:丸善株式会社 
\item
	黒澤馨 尾形わかば 著
	2004
	『電子情報通信レクチャーシリーズ D-8 現代暗号の基礎数理』
	東京:コロナ社
\item
	Schneier,Bruece 著
	山形浩生 監訳
	安達眞弓 新井俊一 緒方理友 katokt 金田忠士 釜池聡太 後藤洋 小宮山亮磨 重国和宏 新山祐介 園田道夫 八田昌三 春山征吾 吉岡雅之 訳
	2003
	『暗号技術大全』
	東京:ソフトバンクパブリッシング
\item
	Krawczyk,H Bellare,M Canetti,R
	IPA訳
	1997
	『HMAC: メッセージ認証のための鍵付ハッシング 』
	東京:独立行政法人 情報処理推進機構
	http://www.ipa.go.jp/security/rfc/RFC2104JA.html
\item
	Dierks,T Allen,C
	IPA訳
	1999
	『TLS プロトコル v1.0』
	東京:独立行政法人 情報処理推進機構
	http://www.ipa.go.jp/security/rfc/RFC2246-00JA.html
\item
	Dierks,T Rescorla,E
	IPA訳
	2008
	『TLS プロトコル v1.2』
	東京:独立行政法人 情報処理推進機構
	https://www.ipa.go.jp/security/rfc/RFC5246-00JA.html
\item
	Rescorla,E
	2008
	TLS Elliptic Curve Cipher Suites with SHA-256/384 and EDS Galois Counter Mode (GCM)
	IETF
	http://datatracker.9ietf.org/doc/rfc5289/
\item
	筆者不詳
	年代不詳
	『X.509電子証明書の問題調査 X.509の解説』
	東京:独立行政法人 情報処理推進機構
	http://www.ipa.jp/security/fy10/contents/over-all/02/21.html
\item
	takeuchi
	年度不詳
	『公開鍵暗号:RSA暗号』
	新潟
	http://www2.cc.niigata-u.ac.jp/~takeuchi/tbasic/BackGround/RSA.html
\item
	Housley,R Ford,W Polk,W Solo,D
	IPA訳
	1999
	『インターネットX.509 PKI - 証明書と CRL のプロファイル』
	東京:独立行政法人 情報処理推進機構
	https://www.ipa.go.jp/security/rfc/RFC2459JA.html	
\item
	CRYPTREC
	2015
	『SSL/TLS 暗号設定ガイドライン』
	東京:独立行政法人 情報処理推進機構
	https://www.ipa.go.jp/files/000045645.pdf
\item
	有田正剛
	2007年
	『公開鍵暗号、電子署名そして鍵共有』
	情報セキュリティ大学院大学
	http://lab.iisec.ac.jp/~arita/pdf/lecture123.pdf
\item
	著者不詳
	2015
	『1024-bit 暗号のその先へ』
	DigiCert
	https://www.digicert.ne.jp/blog/moving-beyond-1024-bit-encryption.html
\end{thebibliography}